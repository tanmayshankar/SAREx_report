%\usepackage{bm}
\chapter{Introduction}

Robots today are capable of a wide variety of tasks, which they achieve by two broad types of methods. The first set of methods involves directly programming the robot to execute a specific task. This form of explicit programming of the robot is suitable for situations in which repetitive tasks are to be performed in known environments with low amounts of uncertainty. 

The second type of method involves the robot learning how to execute the task, or meta-programming. In this case, the robot does not have explicit instructions of exactly what to do, rather it is provided with a general framework that may adapt to different scenarios in its surrounding environment, or uncertainty in its world. 

Incorporating such learning based approaches into robots is crucial for generalizing robots to much wider ranges of applications, or creating versatile robots that can complete a variety of tasks. Some fields and applications that would benefit greatly from general purpose robots include personal robots that assist humans in daily activities, industrial robots required to perform general purpose tasks, and robots that are required to act in collaboration with humans and other robots alike. 

These scenarios call for a high level of understanding on the robot's part, and the ability to act without being explicitly instructed on what to do in a particular situation. Such generalization also allows the robot to handle unforeseen or new circumstances, and thus act more intelligently than previously capable. 

In order to provide these robots with an implicit understanding of the dynamic environments they act in, we turn to a recent by-product of the fusion of computer vision and machine learning, i.e. semantic understanding. 